\documentclass[12pt]{extarticle}
\usepackage{phys440}

\title{PHYS440 - Homework 3}
\author{John Hurst}
\date{May 2024}

\begin{document}
\maketitle

\begin{center}
\noindent\fbox{%
\parbox{0.95\textwidth}{%
Note: As far as I can tell, it is currently not possible to run most circuits created in the IBM Quantum Composer directly on IBM Quantum hardware,
because the Quantum Composer does not support automatic transpilation.
For this reason, I ran all my circuits using Python Qiskit programs, where it is possible to transpile the circuits.
I've included the programs, and images of the original circuits and the transpiled circuits in the submitted files.
}
}
\end{center}

%%%%%%%%%%%%%%%%%%%%%%%%%%%%%%%%%%%%%%%%%%%%%%%%%%%%%%%%%%%%%%%%%%%%%%%%%%%%%%%%%%%%%%%%%%%%%%%%%%%%
\question{1}{Quantum teleportation (10 marks)

Quantum teleportation refers to the ability to transfer a general quantum state $\ket\psi$ between two spatially-separated observers (Alice and Bob) that have previously shared the members of an entangled-qubit pair.
\begin{enumerate}[(a)]
\item Remind yourself how you can create any given qubit state $\ket\psi$ via a unitary transformation $\hat{U}(\theta,\phi)$ acting on $\ket{0}$,
with $\hat{U}(\theta,\phi)$ begin composed of two suitable successive rotations by angles $\theta$ and $\phi$:
\[
\ket{\psi} = \hat{U}(\theta,\phi)\ket{0}
\]
Design a circuit to implement $\ket{\psi}$ for a given set of values for $\theta$ and $\phi$.
\item Implement a quantum-teleportation circuit that has $\ket{\psi}$ from part (a) above as input and also includes a diagnostic to verify that the state $\ket{\psi}$ has indeed been faithfully teleported.
Run the circuit for a representative set of states $\ket{\psi}$ (i.e. angles $\theta$ and $\phi$).
Discuss the quality and magnitude of any observed errors/deviations.
\end{enumerate}
}

A universal single qubit gate $\hat{U}$ that can transform the state $\ket{0}$ to any state $\ket{\psi}$
can be composed of a rotation around the $Y$-axis by an angle $\theta$ followed by a rotation around the $Z$-axis by an angle $\phi$:
\[
\hat{U}(\theta,\phi) = R_Z(\phi)R_Y(\theta)
\]
where
\begin{align*}
R_Y(\theta) &= \begin{pmatrix}
    \cos\frac{\theta}{2} & -\sin\frac{\theta}{2} \\
    \sin\frac{\theta}{2} & \cos\frac{\theta}{2}
\end{pmatrix} \\
R_Z(\phi) &= \begin{pmatrix}
    e^{-i\frac{\phi}{2}} & 0 \\
    0 & e^{i\frac{\phi}{2}}
\end{pmatrix} \\
\hat{U}(\theta,\phi) &= \begin{pmatrix}
    \cos\frac{\theta}{2}e^{-i\frac{\phi}{2}} & -\sin\frac{\theta}{2}e^{i\frac{\phi}{2}} \\
    \sin\frac{\theta}{2}e^{-i\frac{\phi}{2}} & \cos\frac{\theta}{2}e^{i\frac{\phi}{2}}
\end{pmatrix}
\end{align*}

%%%%%%%%%%%%%%%%%%%%%%%%%%%%%%%%%%%%%%%%%%%%%%%%%%%%%%%%%%%%%%%%%%%%%%%%%%%%%%%%%%%%%%%%%%%%%%%%%%%%
\question{2}{Phase estimation: Example implementation (10 marks)

Implement the phase-estimation circuit to find the eigenvalue associated with eigenvector
\[
\ket{u} = \frac{1}{\sqrt{2}}\left(\ket{0} - i\ket{1}\right)
\]
of the unitary matrix
\[
\hat{U}_Y = \begin{pmatrix}
    1 & -i \\
    i & 1
\end{pmatrix}
\]
(\textit{Hint}: Two-bit precision will be enough for the purpose of this task!)
}

%%%%%%%%%%%%%%%%%%%%%%%%%%%%%%%%%%%%%%%%%%%%%%%%%%%%%%%%%%%%%%%%%%%%%%%%%%%%%%%%%%%%%%%%%%%%%%%%%%%%
\question{3}{Application: Quantum computational chemistry (5 bonus marks)

Inform yourself about the application of quantum algorithms (especially phase estimation)
in the context of quantum simulation of physical systems, especially quantum chemistry.
Use and resources at your disposal.
[A useful starting point could be reading some parts of the article S. McArdle et al.\cite{mcardle2020}]
Write a short summary (between $\frac{1}{2}$ and 1 page), discussing the main algorithmic ingredients and the quantum advantage.
}

\printbibliography
\addcontentsline{toc}{section}{References}


\end{document}
