\documentclass[12pt]{extarticle}
\usepackage{phys440}

\title{PHYS440 - Exercises from Nielsen and Chuang (2016)}
\author{John Hurst}
\date{2023/2024}

\begin{document}
\maketitle

% \tableofcontents

%%%%%%%%%%%%%%%%%%%%%%%%%%%%%%%%%%%%%%%%%%%%%%%%%%%%%%%%%%%%%%%%%%%%%%%%%%%%%%%%%%%%%%%%%%%%%%%%%%%%
\question{2.1}{(Linear dependence: example) Show that $(1, -1)$, $(1, 2)$ and $(2, 1)$ are linearly dependent.
}

\[
(1, -1) + (1, 2) = (2, 1)
\]

%%%%%%%%%%%%%%%%%%%%%%%%%%%%%%%%%%%%%%%%%%%%%%%%%%%%%%%%%%%%%%%%%%%%%%%%%%%%%%%%%%%%%%%%%%%%%%%%%%%%
\question{2.2}{(Matrix representations: example) Suppose $V$ is a vector space with basis vectors $\ket{0}$ and $\ket{1}$, and $A$ is a linear operator from $V$ to $V$ such that
$A\ket{0} = \ket{1}$ and $A\ket{1} = \ket{0}$. Find the matrix representation for $A$, with respect to the input basis $\ket{0}, \ket{1}$,
and the output basis $\ket{0}, \ket{1}$.
Find input and output bases which give rise to a different matrix representation of $A$.
}

With respect to the basis $\ket{0}, \ket{1}$, the vector representation of $\ket{0}$ is
$\begin{pmatrix} 1 \\ 0 \end{pmatrix}$ and the vector representation of $\ket{1}$ is $\begin{pmatrix} 0 \\ 1 \end{pmatrix}$.

Let the matrix $A=\begin{pmatrix} 0 & 1\\ 1 & 0\end{pmatrix}$.

Then
\begin{align*}
A\ket{0} & = \begin{pmatrix} 0 & 1\\ 1 & 0\end{pmatrix} \begin{pmatrix} 1 \\ 0 \end{pmatrix} = \begin{pmatrix} 0 \\ 1 \end{pmatrix} = \ket{1} \\
A\ket{1} & = \begin{pmatrix} 0 & 1\\ 1 & 0\end{pmatrix} \begin{pmatrix} 0 \\ 1 \end{pmatrix} = \begin{pmatrix} 1 \\ 0 \end{pmatrix} = \ket{0}
\end{align*}

\hhline

Now consider the basis $\ket{+} = \istwo(\ket{0} + \ket{1})$ and $\ket{-} = \istwo(\ket{0} - \ket{1})$.

Then $\ket{0} = \istwo(\ket{+} + \ket{-})$ and $\ket{1} = \istwo(\ket{+} - \ket{-})$.

So with respect to the basis $\ket{+}, \ket{-}$, the vector representation of $\ket{0}$ is
$\begin{pmatrix} \istwo \\ \istwo \end{pmatrix}$ and the vector representation of $\ket{1}$ is $\begin{pmatrix} \istwo \\ -\istwo \end{pmatrix}$.

Let's denote the matrix representation of the operator $A$ with respect to the basis $\ket{+}, \ket{-}$ as $A'$.

Then
\begin{align*}
A'\begin{pmatrix}\istwo & \istwo \\ \istwo & -\istwo\end{pmatrix} & = \begin{pmatrix} \istwo & \istwo \\ -\istwo & \istwo \end{pmatrix} \\
A' & = \begin{pmatrix} \istwo & \istwo \\ -\istwo & \istwo\end{pmatrix} \inv{\begin{pmatrix}\istwo & \istwo \\ \istwo & -\istwo\end{pmatrix}} \\
& = \begin{pmatrix} 1 & 0 \\ 0 & -1 \end{pmatrix}
\end{align*}

%%%%%%%%%%%%%%%%%%%%%%%%%%%%%%%%%%%%%%%%%%%%%%%%%%%%%%%%%%%%%%%%%%%%%%%%%%%%%%%%%%%%%%%%%%%%%%%%%%%%
\question{2.3}{(Matrix representation for operator products) Suppose $A$ is a linear operator from vector space $V$ to vector space $W$,
and $B$ is a linear operator from vector space $W$ to vector space $X$.
Let $\ket{v_i}$, $\ket{w_j}$ and $\ket{x_k}$ be bases for the vector spaces $V$, $W$ and $X$, respectively.
Show that the matrix representation for the linear transformation $BA$ is the matrix product of the matrix representations for $B$ and $A$, with respect to the appropriate bases.
}

Let $l=\dim V$, $m=\dim W$ and $n=\dim X$.

If we take a vector $\vec{v}\in V$, we can write it as a linear combination of the basis vectors $\ket{v_i}$:
\[
\vec{v} = \sum_{i=1}^l v_i \ket{v_i}
\]
where we interpret $\ket{v_i}$ as a basis vector, and $v_i$ by itself as the coefficient of that basis vector.

Then the matrix representation of the linear operator $A$ is
\[
A\vec{v} = \sum_{j=1}^m \left[\sum_{i=1}^l A_{ji} v_i\right] \ket{w_j}.
\]
Similarly, for the linear operator $B$,
\[
B\vec{w} = \sum_{k=1}^n \left[\sum_{j=1}^m B_{kj} w_j\right] \ket{x_k}.
\]
Therefore, the matrix representation of the linear operator $BA$ is
\begin{align*}
B(A\vec{v}) & = \sum_{k=1}^n \left[\sum_{j=1}^m B_{kj} \left(\sum_{i=1}^l A_{ji} v_i\right)\right] \ket{x_k} \\
& = \sum_{k=1}^n \left[\sum_{j=1}^m \sum_{i=1}^l B_{kj} A_{ji} v_i\right] \ket{x_k} \\
& = \sum_{k=1}^n \left[\sum_{i=1}^l \left(\sum_{j=1}^m B_{kj} A_{ji}\right) v_i\right] \ket{x_k} \\
& = \sum_{k=1}^n \left[\sum_{i=1}^l (BA)_{ki} v_i\right] \ket{x_k} \\
\end{align*}
That is, the matrix representation of the linear operator $BA$ is the matrix product of the matrix representations for $B$ and $A$.

% \printbibliography
% \addcontentsline{toc}{section}{References}

\end{document}
